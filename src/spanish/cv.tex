%!TEX TS-program = xelatex
%!TEX encoding = UTF-8 Unicode
\documentclass[10pt, a4paper]{article}
\usepackage{fontspec} 

% DOCUMENT LAYOUT
\usepackage{geometry} 
\geometry{a4paper, textwidth=5.5in, textheight=8.5in, marginparsep=7pt, marginparwidth=.5in}
\setlength\parindent{0in}

% FONTS
\usepackage{xunicode}
\usepackage{xltxtra}
\defaultfontfeatures{Mapping=tex-text} % converts LaTeX specials (``quotes'' --- dashes etc.) to unicode
\setromanfont [Ligatures={Common}, BoldFont={Fontin Bold}, ItalicFont={Fontin Italic}]{Fontin}
\setsansfont [Ligatures={Common}, BoldFont={Fontin Sans Bold}, ItalicFont={Fontin Sans Italic}]{Fontin Sans}
\setmonofont[Scale=0.8]{Monaco} 
% ---- CUSTOM AMPERSAND
\newcommand{\amper}{{\fontspec[Scale=.95]{Fontin}\selectfont\itshape\&}}
% ---- MARGIN YEARS
\usepackage{marginnote}
\newcommand{\years}[1]{\marginnote{\scriptsize #1}}
\renewcommand*{\raggedleftmarginnote}{}
\setlength{\marginparsep}{7pt}
\reversemarginpar

% HEADINGS
\usepackage{sectsty} 
\usepackage[normalem]{ulem} 
\sectionfont{\rmfamily\mdseries\upshape\Large}
\subsectionfont{\rmfamily\bfseries\upshape\normalsize} 
\subsubsectionfont{\rmfamily\mdseries\upshape\normalsize} 

\usepackage[xetex, bookmarks, colorlinks, breaklinks, pdftitle={CV Gabriel Diegel},pdfauthor={Gabriel Diegel}]{hyperref}  
\hypersetup{linkcolor=blue,citecolor=blue,filecolor=black,urlcolor=blue} 

% DOCUMENT
\begin{document}
{\LARGE Gabriel Diegel}\\[1cm]
Av. Grau 490\\
\texttt{15046} Lima, Perú\\[.2cm]
Fecha de nacimiento: 20 de Diciembre de 1982\\
Nacionalidad: alemán\\[.2cm]
Teléfonos: \texttt{+4917664181625}\\
Correo electrónico: \href{mailto:gabriel.diegel@icloud.com}{gabriel.diegel@icloud.com}\\

%%\hrule
\section*{Última posición}
{\bf Software Engineer}, Tekton, Lima, Perú

%\hrule
\section*{Habilidades}
\subsection*{Lenguajes de programación}
Java, Groovy, Kotlin, Python, R
\subsection*{Tecnologías}
Maven, Spring Boot, Docker, JUnit5, Selenium Webdriver, Protractor, Git, GitlabCI, Jenkins, GoCD, Grafana, Tableau, Ansible, Kubernetes

%\hrule
\section*{Experiencia profesional}
\years{2019-current}{\bf Software Engineer} Tekton, Lima, Peru\\
$\bullet$ Desarollé backend en Spring Boot para cliente peruano\\
$\bullet$ Desarollé conjunto de pruebas de aceptación para frontend y la integración con CI para cliente de EEUU\\
$\bullet$ Desarollé conjunto de pruebas de aceptación para backend para un nuevo CMS para cliente de EEUU\\
$\bullet$ Extendí las herramientas de DevOps para que soporten correr pruebas de aceptación en pipelines de despliegue\\
$\bullet$ Mejoré el marco de pruebas de aceptación introduciendo ejecución asíncrona\\[.2cm]
\years{2018--2019}{\bf Senior Test Engineer} 1\amper{}1 Internet SE, Karlsruhe, Alemania\\
$\bullet$ Desarollé una estrategia modular de pruebas y herramientas para pruebas de frontend basadas en web\\
$\bullet$ Desarollé y apoyé una estrategia y marco para la integración de pruebas de aceptación funcionales en el flujo de trabajo de entraga continua\\
$\bullet$ Planifiqué, ejecuté y analizé pruebas de rendimiento para almacenamiento de mensajes de próxima generación y de un producto Single-Sign-On\\
$\bullet$ Supervisé tesis de estudiantes de postgrado\\
$\bullet$ Dirigí un grupo de trabajo de automatización de pruebas\\[.2cm]
\years{2013--2018}{\bf Test Engineer} 1\amper{}1 Internet SE, Karlsruhe, Alemania\\
$\bullet$ Desarollé conjuntos de pruebas de aceptación en Java para una amplia gama de servicios backend usando protocolos diversos como HTTP, SMTP, IMAP y POP3\\
$\bullet$ Desarollé un marco declarativo en Java para pruebas de API basadas en HTTP\\
$\bullet$ Desarollé un servicio web tipo RESTful y backend para recoger métricas de artefactos de software desplegados\\
\newpage
\years{2010--2013}{\bf Test Analyst} Atos Worldline GmbH, Frankfurt, Alemania\\
$\bullet$ Analizé esfuerzos de prueba, creé planes de pruebas y creé datos de prueba\\
$\bullet$ Planifiqué y ejecuté pruebas de aceptación manuales, asimismo apoyé pruebas de aceptación con clientes externos\\
$\bullet$ Implementé un programa de automatización para pruebas de aceptación funcionales\\
$\bullet$ Desarollé herramientas de CLI con Python para el gestionar datos de prueba extrayendo datos del mainframe, filtrándolos y cargándolos automaticamente en la base de datos Oracle\\
$\bullet$ Analizé errores de sistema y apoyé al equipo de desarollo resolviendo incidentes de producción\\
$\bullet$ Introduje y administré un servidor SVN para el equipo QA\\[.2cm]
\years{2009--2010}{\bf Research Assistant} Fraunhofer IPM, Freiburg, Alemania\\
$\bullet$ Desarollé un software de control para un sistema automatizado de detección de partículas en Matlab\\
$\bullet$ Integré un sistema rápido de adquisición de datos basado en hardware

%\hrule
\section*{Formación Académica}
\years{2009} {\bf Dipl.-Ing. (FH)} en Ingenieria Eléctrica, Hochschule Karlsruhe---Technik und Wirtschaft, Karlsruhe, Alemania\\
Título de tesis: \emph{"Development of an automated image analysis software for a fluorescence--based data acquisition system for the detection of airborne microorganisms"}

%\hrule
\section*{Certificados}
$\bullet$ ISTQB Certified Tester: Foundation Level

%\hrule
\section*{Idiomas}
$\bullet$ alemán: Competencia nativa\\
$\bullet$ español: Competencia profesional de trabajo\\
$\bullet$ inglés: Competencia profesional plena

%\hrule
\section*{Aficiones}
Escribir, viajar y cocinar\\

%\hrule
\section*{Artículos}
\years{2015} \href{https://gitlab.com/gdiegel/writing-public/blob/master/custom\_assertions.md}{"Take your testing DSL to the next level with custom assertions"}\\
\years{2016} \href{https://gitlab.com/gdiegel/writing-public/blob/master/junit5_feeble_screams_from_forests_well_known.md}{"JUnit5--Feeble screams from forests well known"}\\
\years{2017} \href{https://gitlab.com/gdiegel/writing-public/blob/master/neo\_problem\_spring.md}{"Using neo-problem with a pure Spring Boot service"}\\
\years{2018} \href{https://gitlab.com/gdiegel/writing-public/blob/master/custom\_test\_engine\_with\_kotlin.md}{"Writing a custom test engine with Kotlin"}\\
\years{2019} \href{https://gitlab.com/gdiegel/writing-public/blob/master/automated\_acceptance\_testing.md}{"Automated acceptance testing"}\\
\years{2019} \href{https://gitlab.com/gdiegel/writing-public/blob/master/gitlabci\_best\_practices.md}{"Running acceptance tests on GitlabCI"}
\end{document}
